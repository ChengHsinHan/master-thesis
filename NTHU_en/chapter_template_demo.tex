\documentclass[class=NCU_thesis, crop=false]{standalone}
\begin{document}

\chapter{Chapter name(demo)}
Content of chapter \\
Content Content Content.

\section{Section name}
Content of section \\
Content Content Content
\subsection{Subsection name}
Content of subsection \\
Content Content Content

\subsubsection{Subsubsection name}
Content of subsubsection \\
Content Content Content

\paragraph{Paragraph name}
Content of paragraph \\
Content Content Content

\subparagraph{Subparagraph name}
Content of subparagraph \\
Content Content Content


\chapter{Results}

\section{Main Uncertainties}
	The dominant uncertainties in this analysis are the systematic uncertainties. Among them, the b-tagging efficiency, vector boson modeling and $t\bar{t}$ modeling contribute the most.
	
\section{Results}
	A fit to the invariant mass of Higgs candidate $m_{\mathrm{h}}$ is used to search for the signal. For resolved region, $m_{\mathrm{jj}}$ represents the Higgs mass while $m_{\mathrm{J}}$ is used in the merged region.
	
	The fit is based on a likeli-hood based approach. The systematic uncertainties are used in the likeli-hood function as nuisance parameters. The data in SR and two CRs are fit simultaneously for all four different (proxy) $E_T^{\mathrm{miss}}$ bins: $\left[150, 200\right)$ GeV, $\left[200, 350\right)$ GeV, $\left[350, 500\right)$ GeV, and $\left[500, \infty \right)$ GeV. The fit variable used in the one-muon CR is the $\mu$ lepton charge. $t\bar{t}$ processes tend to produce the same amount of $\mu^+$ and $\mu^-$ leptons but the W+jets events produce more $\mu^+$ than $\mu^-$ leptons, which originates from proton-proton collision in LHC and from the conservation of electric charge. $\mu$-charge can thus be made use of as a differentiation of $t\bar{t}$ and W+jets events.


\chapter{Conclusion}


\chapter{Table}
\section{Simple table}
\begin{table}[h]
    \centering
    \caption{Solution}
    \begin{tabular}{| l | l |}
        \hline
        Component  & Concentration(mM) \\ \hline
        \ce{CaCl2} & 118.0 \\ \hline
    \end{tabular}
\end{table}

\section{Auto break line table}
\begin{table}[h]
    \centering
    \begin{tabularx}{\textwidth}{| l | X |}
        \hline
        short & short short \\ \hline
        long  & long long long long long long long long long long \\ \hline
    \end{tabularx}
\end{table}

\end{document}