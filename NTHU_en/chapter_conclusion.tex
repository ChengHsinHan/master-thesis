\documentclass[class=NCU_thesis, crop=false]{standalone}
\begin{document}

\chapter{Event Selection}
A number of selection is applied on the events in this study. Because the final state of the Z'-2HDM model contains no leptons (see fig. \ref{fig:zp2HDM}), one can define this zero-lepton region as the signal region (SR). However, there are other processes having the same final products as that of this model, which is recognized as the background. To further estimate the amount of the background events, one analyzes other processes that are similar to the background. These processes are called the control region (CR). Both SR and CR are discussed in this chapter.

\fig[0.4][fig:zp2HDM][!hbt]{zp2HDM.png}[The Feynman diagram of the Z'-2HDM model.][short caption]

\section{Signal Region}
First, the value of $E_T^{\mathrm{miss}}$ is required to be at least 150 GeV. $\tau$-lepton candidates are rejected. Multijet backgrounds may pass the selection criteria. To further suppress this background, one adds two selections. The first one requires the azimuthal angle between $E_T^{\mathrm{miss}}$ and any of the three highest-$p_T$ jets ($\min(\Delta \phi(E_T^{\mathrm{miss}}, \mathrm{jet}))$) greater than 20$^\circ$. If the number of central jets is less than 3, then the forward jets are only considered. The other selection requires the azimuthal angle between 

\end{document}