\documentclass[class=NCU_thesis, crop=false]{standalone}
\begin{document}

\chapter{Event Selection}
A number of selection is applied on the events in this study. Because the final state of the Z'-2HDM model contains no leptons (see fig. \ref{fig:zp2HDM}), one can define this zero-lepton region as the signal region (SR). However, there are other processes having the same final products as that of this model, which is recognized as the background. To further estimate the amount of the background events, one analyzes other processes that are similar to the background. These processes are called the control region (CR). Both SR and CR are discussed in this chapter.

\fig[0.4][fig:zp2HDM][!hbt]{zp2HDM.png}[The Feynman diagram of the Z'-2HDM model.][short caption]

\section{Signal Region}
	First, no baseline leptons candidates, including the $\tau$-leptons, are considered. The value of $E_T^{\mathrm{miss}}$ is required to be at least 150 GeV. Multijet backgrounds may pass the selection criteria. To further suppress this background, one adds two selections. The first one requires the azimuthal angle between $E_T^{\mathrm{miss}}$ and any of the three highest-$p_T$ jets ($\min(\Delta \phi(E_T^{\mathrm{miss}}, \mathrm{jet}))$) greater than 20$^\circ$. If the number of central jets is less than 3, then the forward jets are only considered. The other selection requires the azimuthal angle between $E_T^{\mathrm{miss}}$ and $p_T^{\mathrm{miss}}$ smaller than 90$^\circ$, where $p_T^{\mathrm{miss}}$ is the negative sum of object momentum measured by the inner detector.

	After these selections, one devides these events into two regions based on the value of $E_T^{\mathrm{miss}}$. Those whose $E_T^{\mathrm{miss}}$ is smaller than 500 GeV are defined as the resolved region. Whilst the merged region contains those events with $E_T^{\mathrm{miss}}$ > 500 GeV. The resolved region are further divided into three regions, whose $E_T^{\mathrm{miss}}$ are within $\left(150, 200\right]$, $\left(200, 350\right]$, and $\left(350, 500\right]$. Different sets of selections are applied in these two regions.

	For the resolved region, the events are required to have at least two small-R jets, which is the reason of the naming of this region. In these jets, exactly two b-tagged jets are also required. The $p_T$ of one of the jets has to exceed 45 GeV. For the events with two (three or more) jets, the scalar sum of the $p_T$ of the highest two (three) jets is also required to be at least 120 (150) GeV. A requirement of $S_{\mathrm{object}} > 16$ is also applies. To further suppress the multijet backgrounds, a selection on the separation are required. Due to the conservation of the momentum, one expects the tracks of the $E_T^{\mathrm{miss}}$ and the Higgs candidate are back-to-back. The azimuthal angle between $E_T^{\mathrm{miss}}$ and the Higgs candidate is required to be greater than 120$^\circ$. The azimuthal angle between the two jets from the Higgs candidate ($\Delta \phi(\mathrm{jet}_1, \mathrm{jet}_2)$)tends to be large in multijet background events because of the topology. Thus, it is required to be smaller than 140$^\circ$. Furthermore, in order to reject t$\mathrm{\bar{t}}$ background events, two more selections are added. One of them is $\Delta R(\mathrm{jet}_1, \mathrm{jet}_2) < 1.8$. The scalar sum of the first three highest $p_T$ of the jets is required to be greater than 63\% of the scalar sum of all jets is considered as the other one.
	
	As for the merged region, the events which contains at least one large-R jets are considered, which is the reason of the name "merged". The large-R (VR track) jet with the highest $p_T$ are referred to as the leading large-R (VR track) jet. Two leading VR track jets associated with the leading large-R jet are required be to b-tagged. Events with any VR track jets outside the large-R jets are rejected. To suppress the t$\mathrm{\bar{t}}$ background events, the value of $p_T$ of the leading large-R jet is required to ecxeed 43\% of the scalar sum of the $p_T$ of the leading large-R jets and all small-R jets.
\end{document}