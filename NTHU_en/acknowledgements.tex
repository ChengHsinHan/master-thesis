\documentclass[class=NCU_thesis, crop=false]{standalone}
\begin{document}

\chapter{Acknowledgement}

首先,我要先感謝我的指導教授,徐百嫻老師。感謝她自我大學三年級始從零開始教導我在這方面的知識與研究態度,也是她在我對於研究內容迷茫時給予我建議與指點。

其次,我也要感謝此次研究的主要領導人,University of Washington的Samuel Meehan博士後研究員、Max-Planck-Institut f\"{u}r Physik的Patrick Reick博士後研究員、Nikhef and Radboud University的Frank Filthaunt教授與University of Washington的Shih-Chieh Hsu教授,感謝他們對於我們的研究提出討論的與統籌。

也感謝Ludwig-Maximilians-Universit\"{a}t M\"{u}nchen的Andrea Matic博士生,她在我剛進到分析團隊時對於我在分析程式理解上有著非常大的幫助。

另外,我也必須感謝呂昀儒博士後研究員,是他提供了在技術上碰到的疑難雜症的處理方式與提供更快速,以及更有系統性的分析方式。

除此之外,也對自己研究上的夥伴,施柏杉同學以及詹予欣學妹在討論上提供的一切幫助,使我能夠透過討論吸取更多知識與處事方法等。

更要銘謝蔡孟儒學弟,雖然其分析的範疇與此有一些差異,但還是願意與我討論我可能碰到的問題,並也提出各種有實質可能性上的建議。

還要對於在分析團隊上對彼此分析有幫助的每一位成員做出最大的感謝,是大家一起的努力才能完成此次的分析成果。

最後,還要感謝所有在背後支持我的人。研究室的羅令崴博士後研究員、簡上淯博士生、李俊豪博士生、葉書瑋博士生、Anand Hegde博士生、陳祐君碩士、盧致融碩士生、王斌碩士生,謝謝他們的陪伴,能在研究遭遇瓶頸之際給予適當歡笑與慰藉。也感謝象棋社的楊宗諭老師、朱緯東學長、郭達毅學長、許浩哲學弟、楊上民學弟、陳其伸學弟能讓我在每週有短暫的時光得以釋放壓力。更感謝在研究之餘參與助教工作認識的詹貴麟博士生與孫乙立碩士,謝謝你們經常與我交換研究生涯碰到的意見。

要感謝的人不僅止於此,筆者一併在此再一由衷感謝,未能盡到周詳提及之處,尚且包涵。

\end{document}


