\documentclass[class=NCU_thesis, crop=false]{standalone}
\begin{document}

\chapter{Object Selection and Reconstruction}
	Signals recorded in the ATLAS detector are categorized or reconstructed as pysical objects, which could be further used in the analyses. The reconstruction and definition of objects used in this study are listed in the following.
	
\section{Jets}	
	
\section{Leptons}
	Leptons are used to categorized the regions selected in this analysis, which would be covered in the next chapter. Requirements of each flavor are explained in the following and summarized in table.\ref{tab:lepton selection}.
	
	\begin{table}[h]
	%\centering
	\caption{The summarization of lepton selection and reconstruction. The rightmost column are the requirements for the reconstructed small-R jet that decays from a $\tau$-lepton candidate.}
	\label{tab:lepton selection}		\begin{tabular}{|c|c|c|c|c|c|}
		\hline
		flavor & \multicolumn{2}{c|}{e} & \multicolumn{2}{c|}{$\mu$} & $\tau$ \\ \hline
		categorization & baseline & signal & baseline & signal & - \\ \hline
		$p_T$ (GeV) & > 7 & > 27 & > 7 & > 25 & > 20 \\ \hline
		$\lvert \eta \rvert$ & \multicolumn{2}{c|}{(0, 2.47)} & (0, 2.7) & (0, 2.5) & (0, 1.37) $\cup$ (1.52, 2.5) \\ \hline
		ID & \multicolumn{2}{c|}{Loose} & \multicolumn{2}{c|}{\shortstack{Loose (0/2-muon)\\ Medium (1-muon)}} & Loose \\ \hline
		transverse impact parameter & \multicolumn{2}{c|}{$d_0 / \sigma(d_0) < 5$} & \multicolumn{2}{c|}{$d_0 / \sigma(d_0) < 3$} & $d_0 < 1$ mm \\ \hline
		$\lvert z_0 \sin(\theta) \rvert$ (mm) & \multicolumn{4}{c|}{< 0.5} & < 1.5 \\ \hline
		Additional & \multicolumn{4}{c|}{-} & \shortstack{one to four track-jets\\$\Delta \phi(\tau, \slashed{E}_T) < \frac{\pi}{8}$}\\ \hline
		\end{tabular}
	\end{table}

	\subsection{Electrons}
		Electron candidates are reconstructed from the energy deposits in EM calorimeter that match a track recorded in the inner detector. In addition, there is a likelihood-based (LH) algorithm, which further makes use of multivariable analysis (MVA), applied for the electron ID. Three levels of ID operating point, loose, medium and tight, are povided; among them, loose ID is used in this study for the electrons. Moreover, electrons are divided into two groups, the baseline electrons, whose transverse momentum, or $p_T$, exceed 7 GeV, and the signal electrons, which requires a tighter threshold of $p_T > 27$ GeV. All candidates within $\lvert \eta \rvert < 2.47$ are considered. Finally, requirements of the impact parameter are considered, both in the transverse and logitudinal directions. For the former, the relative resolution, which is the fraction of the transverse impact parameter $d_0$ and its resolution $\sigma(d_0)$, has an upper bound of 5. For the latter, the value $\lvert z_0 \sin(\theta) \rvert < 0.5$ mm is set, where $z_0$ is the point closet to the vertex along the logitudinal axis and $\theta$ is the polar angle of the track.
		
	\subsection{Muons}
		Muon candidates, also divided into baseline and signal muons, are reconstructed with high dependence of inner detector and the muon spectrometer. Signal muons, whose $p_T$ exceed 27 Gev, are more likely to leave tracks in the inner detector, and thus shall be found in $\lvert \eta \rvert < 2.5$. For baseline muons, which have a looser $p_T$ threshold of 7 GeV, leaving signals in the inner detector is not required and thus are within $\lvert \eta \rvert < 2.7$, which is the range of the muon spectrometer. On top of that, the impact parameter must be consistent with the reconstructed location of the collision, or the primary vertex. $d_0 / \sigma(d_0) < 3$ and $\lvert z_0 \sin(\theta) \rvert < 0.5$ mm are set. Finally, a loose ID is used for the zero-lepton and two-muon channel whilst the one-muon channel makes use of a medium ID for the muons in this analysis. These channels would be covered in the upcoming chapter.
		
	\subsection{Taus}
		$\tau$-leptons, whose decay length is few $\mu$m, barely reach the ATLAS detector and thus are mainly reconstructed from their decay products. Due to the fact that most of the $\tau$-leptons decay into hadrons, the $\tau$-lepton candidates are reconstructed from jets. The transverse impact parameter, $\lvert d_0 \rvert < 1$ mm, and the logitudinal one, $\lvert z_0 \sin(\theta) \rvert < 1.5$ mm, are set for the jet tracks and the $\tau$ vertex. The threshold on the $p_T$ of the jets is set at 20 GeV; the range of $\lvert \eta \rvert < 2.5$, excluding $1.37 < \lvert \eta \rvert < 1.52$, which is the region between the barrel and the forward region, or the crack region, is also required. Furthermore, the ID is built on a boosted decision tree (BDT) that makes use of the information from the tracks and the calorimeter. The loose working point on the $\tau$-leptons is used. Finally, the small-R jet is required to contain one to four track-jets and within a range of $\Delta \phi < \frac{\pi}{8}$ with the missing transverse energy (MET, $E_T^{miss}$ or $\slashed{E}_T$) in order to suppress the W-boson-decayed $\tau$-leptons.
	
\section{Missing Transverse Energy}
	
\section{Overlap Removal}

\end{document}