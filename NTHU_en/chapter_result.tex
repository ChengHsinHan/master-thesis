\documentclass[class=NCU_thesis, crop=false]{standalone}
\begin{document}

\chapter{Object Selection and Reconstruction}
	Signals recorded in the ATLAS detector are categorized or reconstructed as pysical objects, which could be further used in the analyses. The reconstruction and definition of objects used in this study are listed in the following.
	
\section{Leptons}
	\subsection{Electrons}
		Electron candidates are reconstructed from the energy deposits in EM calorimeter that match a track recorded in the inner detector. In addition, there is a likelihood-based (LH) algorithm, which further makes use of multivariable analysis (MVA), applied for the electron identification (ID). Three levels of ID operating point, \textit{loose}, \textit{medium} and \textit{tight}, are povided for electrons. \textit{Loose} ID and criteria is used in this study. Moreover, electrons are divided into two groups, the baseline electrons, whose transverse momentum ($p_T$) exceeds $7$ GeV, and the signal electrons, which requires a tighter threshold of $p_T > 27$ GeV. Finally, all candidates within $\lvert \eta \rvert < 2.47$ are considered.
		
	\subsection{Muons}
		

\section{Jets}
	
\section{Missing Transverse Energy}
	
\section{Overlap Removal}

\end{document}