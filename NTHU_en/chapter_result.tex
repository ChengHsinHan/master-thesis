\documentclass[class=NCU_thesis, crop=false]{standalone}
\begin{document}

\chapter{Object Selection and Reconstruction}
	Signals recorded in the ATLAS detector are categorized or reconstructed as pysical objects, which could be further used in the analyses. Besides the head-to-head (or hard-scattered) events which have high transverse momentum ($p_T$) there are also additional collisions with lower $p_T$. These are referred to as the pile-up events, which one often wants to exclude in the analyes. The reconstruction and definition of objects used in this study are listed in the following.
	
\section{Jets}\label{jet}
	Jets are the reconstruction of collimated bunches of hadrons. In ATLAS experiment, the anti-$k_T$ algorithm is usually used to recombine hadrons into cone-sized shapes. In short, the anti-$k_T$ algorithm makes use of $p_T$ of each given entities and reconstructs jets whose R is the given radius parameter and which center on hard-scattered particles. How the jets are reconstructed and defined in this study is tabulated in table.\ref{tab:jet selection} and explained as follows.
	
	\begin{table}[h]
		\centering
		\caption{Summary of the jet selection and reconstruction.}
		\label{tab:jet selection}
		\begin{tabular}{|c|c|c|c|c|}
			\hline
			type & \shortstack{(\textit{central})\\small-R jets} & \shortstack{(\textit{forward})\\small-R jets} & large-R jets & VR track jets \\ \hline
			$p_T$ (GeV) & > 20 & > 30 & > 200 & > 0.5 \\ \hline
			$\lvert \eta \rvert$ & (0, 2.5) & (2.5, 4.5) & (0, 2.0) & (0, 2.5) \\ \hline
			$R$ & \multicolumn{2}{c|}{0.4} & 1.0 & \shortstack{$\rho / p_T \in$ (0.02, 0.4)\\ with $\rho = 30$ GeV} \\ \hline
			additional & \shortstack{if $\lvert \eta \rvert < 2.4$ then\\$p_T < 60$ GeV} & \multicolumn{2}{c|}{-} & $\lvert z_0 \sin(\theta) \rvert < 3$ mm \\ \hline
		\end{tabular}
	\end{table}
	
	\subsection{Small-radius Jets}
		Small-radius (small-R) jets have a relative smaller size compared to large-radius (large-R) jets, which would be discussed in subsection \ref{large-R jets}. They are main feature when the missing transverse momentum ($E_T^{\mathrm{miss}}$, discussed in section \ref{MET}) is below a certain value (see section \ref{SR}).
	
		Using the anti-$k_T$ algorithm with a raidus parameter of 0.4 small-R jets can be reconstructed from the energy deposits in the calorimeter. Small-R jets are further categorzied into two types, the central jets and the forward jets. Central small-R jets are those within $\lvert \eta \rvert < 2.5$ and whose $p_T > 20$ GeV. Forward small-R jets are within $2.5 < \lvert \eta \rvert < 4.5$ and are required of a $p_T$ threshold of greater than 30 GeV. For jets in $\lvert \eta \rvert < 2.4$, an additional threshold of $p_T < 60$ GeV is require. To further suppress these jets from the pile-up interactions, they are required to be originated from the reconstructed location of the collision, or the primary vertex. In addition, jets containing b-quark are referred to as b-jets. The method used to identify them as b-jets are called b-tagging. Because b-quarks have a relative longer lifetime compared to other particles, they would emit from a secondary vertex rather than the primary one, which can be made use of to perform the b-tagging. A working point of 77\% on averge for the b-tagging efficiency is used in this analysis.
	
	\subsection{Large-radius Jets}\label{large-R jets}
		Large-R jets have a relative larger size compared to small-R jets, which is covered in the last subsection. They are the characteristic objects when the $E_T^{\mathrm{miss}}$ exceeds a threshold value (see section \ref{SR}). Usually they are associated with variable-radius (VR) jets, which will be covered in subsection \ref{VR jets}.
		
		The large-R jets are reconstructed via the anti-$k_T$ algorithm with a radius parameter of 1.0. The reconstruction highly depends on the calorimeter and the tracking system. Certain calibration method is appiled for the energy deposits in the calorimeter. The calibrated jets have a $p_T$ threshold of 200 GeV and are required to be within the range of $\lvert \eta \rvert < 2.0$.
	
	\subsection{Variable-radius Jets}\label{VR jets}
		The large-R jets are a result of two "merged" small jets. In order to identify the small jets inside the reconstructed large-R jets, the VR track jets are introduced.
	
		The VR track jets are also reconstructed using the anti-$k_T$ algorithm. Those jets whose $p_T$ does not exceed 0.5 GeV or out of the range of $\lvert \eta \rvert < 2.5$ are not considered. To suppress the pile-up jets, a creiteria on the longitudinal impact parameter, $\lvert z_0 \sin(\theta) \rvert < 3$ mm, is required. In this, $z_0$ is the point closet to the vertex along the logitudinal axis and $\theta$ is the polar angle of the track. The main feature of the VR track jets is that the radius parameter depends on the value of $p_T$ rather than a constant value:
		\begin{equation}
			R \rightarrow R_{\mathrm{eff}}(p_T) \approx \frac{\rho}{p_T},
		\end{equation}
		in which $\rho$ is set at 30 GeV, which is the optimal value ragarding the efficiency for b-tagging. The upper limit and lower limit, $R_{\mathrm{max}}$ and $R_{\mathrm{min}}$ are set at 0.4 and 0.02 in this regard.
	
\section{Leptons}
	Leptons are used to categorized the regions selected in this analysis, which would be covered in chapter \ref{Event selection}. Requirements of each flavor are explained in the following and summarized in table.\ref{tab:lepton selection}.
	
	\begin{table}[h]
	%\centering
	\caption{Summary of lepton selection and reconstruction. The rightmost column are the requirements for the reconstructed small-R jet that decays from a $\tau$-lepton candidate. The parentheses in the cell of $\mu$ ID would be covered in the next chapter.}
	\label{tab:lepton selection}		\begin{tabular}{|c|c|c|c|c|c|}
		\hline
		flavor & \multicolumn{2}{c|}{e} & \multicolumn{2}{c|}{$\mu$} & $\tau$ \\ \hline
		categorization & baseline & signal & baseline & signal & - \\ \hline
		$p_T$ (GeV) & > 7 & > 27 & > 7 & > 25 & > 20 \\ \hline
		$\lvert \eta \rvert$ & \multicolumn{2}{c|}{(0, 2.47)} & (0, 2.7) & (0, 2.5) & (0, 1.37) $\cup$ (1.52, 2.5) \\ \hline
		ID & \multicolumn{2}{c|}{Loose} & \multicolumn{2}{c|}{\shortstack{Loose (0/2-lepton)\\ Medium (1-muon)}} & Loose \\ \hline
		transverse impact parameter & \multicolumn{2}{c|}{$d_0 / \sigma(d_0) < 5$} & \multicolumn{2}{c|}{$d_0 / \sigma(d_0) < 3$} & $\lvert d_0 \rvert < 1$ mm \\ \hline
		$\lvert z_0 \sin(\theta) \rvert$ (mm) & \multicolumn{4}{c|}{< 0.5} & < 1.5 \\ \hline
		Additional & \multicolumn{4}{c|}{-} & \shortstack{one to four track-jets\\$\Delta \phi(\tau, E_T^{\mathrm{miss}}) < \frac{\pi}{8}$}\\ \hline
		\end{tabular}
	\end{table}

	\subsection{Electrons}
		Electron candidates are reconstructed from the energy deposits in EM calorimeter that match a track recorded in the inner detector. All candidates within $\lvert \eta \rvert < 2.47$ are considered. In addition, there is a LLH-based algorithm, which further makes use of multivariable analysis (MVA), applied for the electron ID. Three levels of ID operating point, loose, medium and tight, are povided; among them, loose ID is used in this study for the electrons. Requirements of the impact parameter are examined, both in the transverse and logitudinal directions. For the former, the relative resolution, which is the fraction of the transverse impact parameter $d_0$ and its resolution $\sigma(d_0)$, has an upper bound of 5. For the latter, the value $\lvert z_0 \sin(\theta) \rvert < 0.5$ mm is set.
		
		Electrons are divided into two groups. The baseline electrons are those whose $p_T$ exceed 7 GeV. The signal electrons, which require a tighter threshold of $p_T > 27$ GeV, are the other groups. The purpose of this categorization is to label which electrons decay more likely from heavier particles. Therefore, it can be used as a sign of different analysis regions. More details of these regions can be found in chapter \ref{Event selection}.
		
	\subsection{Muons}
		Muon candidates, also divided into baseline and signal muons, are reconstructed with high dependence of inner detector and the muon spectrometer. The purpose of the categorization is the same as that of the electrons. Signal muons are those whose $p_T$ exceed 27 Gev. Also, one needs to have more acurate identification of the signal muons compared to the baseline ones. They are required to leave tracks in the inner detector, and thus shall be found in $\lvert \eta \rvert < 2.5$. For baseline muons, which have a looser $p_T$ threshold of 7 GeV, an acurate identification is not sufficient. Therefore, leaving signals in the inner detector is not required. Their pseudorapidity range are set within $\lvert \eta \rvert < 2.7$, which is the range of the muon spectrometer.
		
		On top of that, the impact parameter must be consistent with the the primary vertex. $d_0 / \sigma(d_0) < 3$ and $\lvert z_0 \sin(\theta) \rvert < 0.5$ mm are set. Finally, a loose ID is used for the zero-lepton and two-lepton channel whilst the one-muon channel makes use of a medium ID for the muons in this analysis. These channels would be covered in the upcoming chapter.
		
	\subsection{Taus}
		$\tau$-leptons, whose decay length is few $\mu$m, barely reach the ATLAS detector and thus are mainly reconstructed from their decay products. Due to the fact that most of the $\tau$-leptons decay into hadrons, the $\tau$-lepton candidates are reconstructed from jets. The transverse impact parameter, $\lvert d_0 \rvert < 1$ mm, and the logitudinal one, $\lvert z_0 \sin(\theta) \rvert < 1.5$ mm, are set for the jet tracks and the $\tau$ vertex. The threshold on the $p_T$ of the jets is set at 20 GeV; the range of $\lvert \eta \rvert < 2.5$, excluding $1.37 < \lvert \eta \rvert < 1.52$. The excluded region is those between the barrel and the forward region, or the crack region. Furthermore, the ID is built on a boosted decision tree (BDT) that makes use of the information from the tracks and the calorimeter. The loose working point on the $\tau$-leptons is used. Finally, the small-R jet is required to contain one to four track-jets and within a range of $\Delta \phi < \frac{\pi}{8}$ with the missing transverse energy ($E_T^{\mathrm{miss}}$, see section \ref{MET}) in order to suppress the W-boson-decayed $\tau$-leptons.
	
\section{Missing Transverse Momentum}\label{MET}
	The missing transverse momentum is the imbalance in $p_T$ of the reconstructed objects. Thus, it can be reconstructed as the negative vector transverse momentum sum. This includes the hard-scattered objects and the soft term. The momentum of hard-scattered reconstructed particles construct the former. As for the soft term, tracks that are not associated to any reconstructed hard objects are considered. The vetoed leptons, which are the leptons that needs to removed from the analysis in some channels, which would be covered in the next chapter, also count as the soft term. Jets that failed the selection of small-R jets also contributes the soft term, with the $p_T$ threshold on the forward small-R jet being modified to 20 GeV. The absolute value of the missing transverse momentum is denoted as $E_T^{\mathrm{miss}}$.
	
	An estimation variable $E_T^{\mathrm{miss}}$ significance, S, is used to check the genuinity in whether $E_T^{\mathrm{miss}}$ comes from undectable particles instead of mismeasurements or any inefficiencies. There are two types of $E_T^{\mathrm{miss}}$ significance, the event-based one and the object-based one. The former one is the ratio of the $E_T^{\mathrm{miss}}$ and the square root of the transverse momentum of hard-scattered particles, $H_T$, i.e.,
	\begin{equation}
		S_{\mathrm{event}} = \frac{E_T^{\mathrm{miss}}}{\sqrt{H_T}},
	\end{equation}
	where
	\begin{equation}
		H_T = \Sigma p_T^{\mu} + \Sigma p_T^e + \Sigma p_T^{\gamma} + \Sigma p_T^{\tau} + p_T^{\mathrm{jets}}.
	\end{equation}
	If the soft term contributes little in the transverse momentum, the value of $H_T$ would be approximately that of $E_T^{\mathrm{miss}}$, and thus the value of $E_T^{\mathrm{miss}}$ significance tends to be small. It would be used in the background estimation mainly and would be used as cut in the next chapter.
	
	The other significance value, the object-based one, is a log-likelihood ratio value. It is used to test the hypothesis that the total momentum of the invisible particle is zero against the hypothesis that it is not. In short, if the value of the object-based significance, $S_{\mathrm{object}}$ which is non-negative, is close to zero, then it implies high possibilities that $E_T^{\mathrm{miss}}$ does not come from the invisible particles; if the value is large, then it implies a possibility of existing invisible particles.
	
\section{Overlap Removal}
	Object ambiguities happen when objects match multiple reconstruction criteria. In the following listed specific steps of object reconstruction that are required to solve the problem.
	\begin{enumerate}
		\item If two electron candidates share the same track, remove the one with lower $p_T$.
		\item If a $\tau$-lepton candidate lies within $\Delta R = 0.2$ of an electron or muon, it is removed.
		\item Reject the electron candidates whose track is shared with a muon candidate.
		\item Small-R jets are removed if they are within $\Delta R = 0.2$ of an electron.
		\item Remove the electrons that are within $\Delta R = \min(0.4, 0.04 + 10 \mathrm{GeV} / p_T^{\mathrm{electron}})$ of a small-R jet.
		\item If the separation between a small-R jet and a muon is within $\Delta R = 0.2$, the small-R jet is removed provided that it has fewer than three tracks or that the muon $p_T$ is greater than 50\% of the jet $p_T$ and is greater than 70\% of the $p_T$ sum of the tracks associated to the jets.
		\item Remove the muons that are within $\Delta R = \min(0.4, 0.04 + 10 \mathrm{GeV} / p_T^{\mathrm{muon}})$ of a small-R jet.
		\item Large-R jets whose track is within $\Delta R = 0.1$ to that of an electron are removed.
	\end{enumerate}

\end{document}