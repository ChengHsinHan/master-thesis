\documentclass[class=NCU_thesis, crop=false]{standalone}
\begin{document}


\chapter{Introduction}
	It is generally acknowledged that the universe does not only contain matter. Dark matter (DM) comprises more percentage in the universe than the matter, and its nature remains one of the biggest unknown question in physics. A striking hypothesis on the nature of DM is that it is a electrically neutral, and thus being dark, stable particle, denoted as $\chi$. It has weak interactions with Standard Model (SM) particles in addition to gravitational interactions. Its mass is predicted be to range from the order of GeV to TeV. In this hypothesis, ways of its detection are presented and carried out as shown in figure \ref{fig:DM search}. A collider way of detection for the DM is discussed in this thesis.

	\fig[0.4][fig:DM search][!hbt]{DM_search.png}[Different ways of dark matter detection. Studies on the SM particles that predicted to be the products of DM interactions are the indirect search. Researches on the SM particles which are assumed to have interaction with DM are the direct way of search. Colliding the SM particles and investigating the undetected parts of the products is the collider way of detection.][short caption]

	\newpage

	The DM signature in the collider under the hypothesis that they weakly interact with SM particles is the missing momentum. It can be observed by a detectable particle \textit{X} that produced in associated with the DM. Such \textit{X}+missing momentum signature is explored by the LHC (Large Hadron Collider) experiments with \textit{X} being a jet (would be discussed in section \ref{jet}), a heavy quark, a vector boson, or a Higgs boson. This thesis presents the search with X being a Higgs boson \textit{h} that further decays into a pair of b-quarks, $h \rightarrow b\bar{b}$, which is the most frequent decay channel of \textit{h}. The analysed dataset or proton-proton collisions has the center-of-mass energy $\sqrt{s} = 13$ TeV and was recorded by ATLAS (\textbf{A} \textbf{T}oroidal \textbf{L}HC \textbf{A}pparatu\textbf{S}, would be discussed in chapter \ref{ATLAS}) during 2015 to 2017, corresponding to an integrated luminosity of 79.8 $fb^{-1}$.

	Another progress in this analysis is the introduced variable-radius jet, which would result in improvements in object reconstruction and performance. More details would be discussed in section \ref{VR jets} and its effect compared to the previous iteration is covered in chapter \ref{result}.
	
	The siganl model used in this iteration is a Type-II two-Higgs-doublet model (2HDM) with an additional mediator Z', which is referred to as the Z'-2HDM model. In the siganl model, one scalar \textit{h} Higgs boson and a pseudo-scalar Higgs \textit{A} is found. The whole process reads $pp \rightarrow Z' \rightarrow Ah \rightarrow \chi \bar{\chi} b \bar{b}$, in which the DM pair decay from \textit{A} and the b-quarks pair decay from \textit{h}. The Feynman diagram is shown in figure \ref{fig:zp2HDM}. The relevant parameters are the masses $m_{\mathrm{A}}$, $m_{\mathrm{Z'}}$, $m_{\chi}$, the coupling of Z', $g_{Z'}$ and $\tan(\beta)$, the ratio of the vacuum expectation values of the two Higgs fields coupling to up-type and down-type quarks.
	
	\fig[0.4][fig:zp2HDM][!hbt]{zp2HDM.png}[The Feynman diagram of the Z'-2HDM model.][short caption]
	
	The main SM backgrounds are the top-quark pairs ($t\bar{t}$) and vector bosons (W or Z boson) with b-jets. The signal region and control region are used to constrain the background. More details would be covered in chapter \ref{Event selection}.

\end{document}