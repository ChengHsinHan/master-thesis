\documentclass[class=NCU_thesis, crop=false]{standalone}
\begin{document}

\chapter{The ATLAS detector}
\section{Coordinates}
	The ATLAS (\textbf{A} \textbf{T}oroidal \textbf{L}HC \textbf{A}pparatu\textbf{S}) experiment is one of the seven detector in Large Hadron Collider (LHC) at CERN (European Organization for Nuclear Research). Its cylindrical symmetry and end caps covers nearly $4\pi$ in solid angle.
	
	A coordinate system is used to describe every recorded signals nearby. The origin is set at the center of the detector, or the interaction point (IP). The x-axis points toward the center of the LHC ring; the y-axis points vertically upward; the z-axis points along one of the beam pipe direction such that a right-handed coordinate sysetem is created.
	
	A modified version of cylindrical coordinate is more commonly used in the experiment. The pseudorapidity $\eta \equiv -\ln\tan(\theta / 2)$, in which $\theta$ is the polar angle in cylindrical coordinate, is used to decribe the angle between the z-axis and the direction of interest. $(r, \phi)$ is the same system to describe the tranverse plane, with $\phi$ being the azimuthal angle. In addition, the cone size is defined as $\Delta R \equiv \sqrt{(\Delta \phi)^2 + (\Delta \eta)^2}$.

\section{Components of ATLAS}
	Depending its function, the components are categorized into four parts - inner detector, calorimeter, muon detector, and the magnetic system. Each of them consist of smaller layers.
	
	\subsection{Inner Detector}
		Beginning few centimeters from the IP, the inner decector's main function is to track the trace of charged particles by their interations with the materials. A 2T magnetic field, which surrounds the whole inner detector, causes the charged ones to bend. Based on the directions and the curvatures, one can determine their charges and momenta preliminarily. The innder detector comprises three parts - the pixel detector, the semi-conductor tracker (SCT), and the transition rediation tracker.
		
		Located at the innermost part, the pixel detector contains three layers of modules, which is made up of 250 $\mu$m-thick silicon, each is 2 centimeters by 6 centimeters in size in the direction perpendicular to the beam. Three disks, which are made up of similar material, are at each end cap of the detector. Each module includes about 47,000 pixels, measuring 50 by 400 $\mu$m each. It covers pseudorapidity range $\lvert \eta \rvert < 2.5$ and its proximity to the IP is meant to measure extremely precise trace of the charged particles.
		
		The semi-conductor tracker, having a similar concept and function to the pixel detector, lies in the middle part of the inner detector. Although having a resemblance to the pixel detector, the SCT is in a long and narrow strip-shape rather than small pixels and covers the perpendicular directions to the beam instead of nearly full coverage. The SCT, which overlays a larger area than that of the pixel detector, has more sampled points and thus can track 
	
	\subsection{Calorimeter}
	
	\subsection{Muon Detector}

\end{document}