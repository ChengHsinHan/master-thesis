\documentclass[class=NCU_thesis, crop=false]{standalone}
\begin{document}

\chapter{The ATLAS detector}
\section{Coordinates}
	The ATLAS (\textbf{A} \textbf{T}oroidal \textbf{L}HC \textbf{A}pparatu\textbf{S}) experiment is one of the seven detector in Large Hadron Collider (LHC) at CERN (European Organization for Nuclear Research). Its cylindrical symmetry and end caps covers nearly $4\pi$ in solid angle.
	
	A coordinate system is used to describe every location near ATLAS. The origin is set at the center of the detector, or the interaction point. The x-axis points toward the center of the LHC ring; the y-axis points vertically upward; the z-axis points along one of the beam pipe direction such that a right-handed coordinate sysetem is created.
	
	A modified version of cylindrical coordinate is more commonly used in the experiment. The pseudorapidity $\eta \equiv -\ln\tan(\theta / 2)$, in which $\theta$ is the polar angle in cylindrical coordinate, is used to decribe the angle between the z-axis and the direction of interest. $(r, \phi)$ is the same system to describe the tranverse plane, with $\phi$ being the azimuthal angle. In addition, the cone size is defined as $\Delta R \equiv \sqrt{(\Delta \phi)^2 + (\Delta \eta)^2}$.

\end{document}